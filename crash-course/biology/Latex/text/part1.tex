\hypertarget{carbonne}{%
\chapter{Carbonne}\label{carbonne}}

Ici nous allons voir ce que sont les liaisons covalentes, ioniques et
les ponts hydrogène, ensuite les couches électroniques et de la règle de
l'octet. Mais qu'est-ce que tout ça a à voir avec un homme fou appelé
Gilbert Lewis ? Nous allons y revenir.

Comme toute bonne chanson de 50 Cent, la biologie c'est juste à propos
de sexe et de vie éternelle. Toutes les personnes qui existent devraient
être intéressées par le sexe et la vie éternelle si on considère que
vous êtes, je présume, un être humain.

\hypertarget{le-carbone-un-atome-polygame}{%
\section{Le carbone : un atome polygame
?}\label{le-carbone-un-atome-polygame}}

Les composés organiques sont une classe de composés qui contiennent du
carbone. Et le carbone est assez "inintéressée par la monogamie", pour
dire simplement. Le carbone est en fait assez petit c'est a dire que
pour un atome, il est relativement petit. Il a 6 protons et 6 neutrons
pour une masse atomique totale de 12. À cause de ça, le carbone ne prend
pas beaucoup de place. Et donc le carbone peut former des anneaux, des 
feuilles, des spirales des liaisons doubles voire triples.
Il peut faire pas mal de choses que des atomes plus volumineux ne
sauraient jamais faire. C'est en quelque sorte, l'équivalent atomique
d'une gymnaste olympique. Et il peut faire toutes ces choses
merveilleuses, belles et élégantes parce qu'il est minuscule. On dit
aussi que le carbone est "gentil". Et c'est une chose intéressante à
dire à propos d'un atome. Il n'est pas comme d'autres éléments qui
essayent désespérément de remplir leurs couches électroniques, non, le
carbone sait ce que c'est d'être seul, et il n'est pas en manque
"d'affection" comme le sont le fluor, le chlore ou le sodium. Des éléments
comme le chlore, si vous les respirez, ils vont littéralement vous
déchirer de l'intérieur. Et le sodium, si vous le mettez dans de l'eau,
il explose ! Le carbone, veut, en effet, plus d'électrons, mais il ne va
pas "tuer" pour les obtenir. Il fait et casse des liaisons assez
faiblement.

Le carbone est aussi, comme dit plus haut, inintéressée par la
monogamie, parce qu'il a besoin de 4 électrons supplémentaires et donc
il va se lier avec quasi tout ce qu'il va trouver à portée de main. Et
également parce qu'il a besoin de 4 électrons, il va se lier avec 2, ou
3 ou même 4 de ces choses en même temps. Le carbone est intéressé de se
lier avec beaucoup d'autres molécules comme l'hydrogène, l'oxygène, le
phosphore, l'azote ou d'autres molécules de carbone. Il peut faire ça
dans des configurations infinies ce qui lui permet d'être l'atome
central de structures compliquées qui composent les choses vivantes
telles que nous-même. Parce que le carbone est petit, gentil et
polygame, la vie repose entièrement sur cet élément.

Le carbone est le fondement de la biologie. C'est tellement fondamental
que les scientifiques ont parfois difficile de se représenter une vie
qui ne soit pas basée sur le carbone. La vie est seulement possible sur
terre parce que le carbone est toujours en train de flotter dans notre
atmosphère sous forme de dioxyde de carbone. Donc c'est important de
noter, quand on parle du carbone qui se lie avec d'autres éléments.

\hypertarget{les-couches-electroniques}{%
\section{Les couches electroniques}\label{les-couches-electroniques}}

Le carbone tout seul, est un atome avec 6 protons, 6 neutrons et 6
électrons. Les atomes ont des couches électroniques, et ils ont besoin
qu'elles soient remplies pour être des atomes heureux et accomplis. Donc
le carbone a un total de 6 électrons, 2 sur la première couche pour
qu'il soit tout à fait content et 4 des 8 dont il a besoin pour remplir
la seconde couche. Le carbone forme une sorte de liaison que nous
appelons covalente. Ca arrive quand des atomes partagent des électrons
les uns avec les autres. Donc dans le cas du méthane, qui est en quelque
sorte le composé de carbone le plus simple. Le carbone partage les 4
électrons de sa dernière couche électronique avec 4 atomes d'hydrogène.
Les hydrogène n'ont qu'un électrons, donc ils veulent que leur première
couche S soit remplie. Le carbone partage ses 4 électrons avec ces 4
hydrogènes. Et ces 4 hydrogènes, partagent chacun un électron avec le
carbone. Donc tout le monde est content.

En chimie et en biologie, c'est souvent symbolisé par ce qu'on appelle
la représentation de Lewis.

Gilbert Lewis, la personne qui a inventé la représentation de Lewis,
était aussi derrière les acides et les bases de Lewis et il a été nominé
pour le prix Nobel 35 fois. C'est plus de nominations que n'importe qui
dans l'histoire. Et le nombre de fois où il a gagné est plus ou moins
égale au nombre de fois que n'importe qui a gagné. Ce qui est
zéro\ldots{} Lewis détestait ça. C'est un peu comme un joueur de
baseball qui a plus de touches que tout le monde dans l'histoire mais
aucun home run. Il se peut que Lewis ait été le chimiste le plus
influençant de tous les temps. Il a inventé le terme ``photon'', il a
révolutionné notre façon de voir les acides et les bases et il a produit
la première molécule d'eau lourde. Il a été la première personne à
conceptualiser la liaison convalente dont on parle maintenant. Gilbert
Lewis est mort seul dans son laboratoire, pendant qu'il travaillait sur
des composés de cyanure après avoir été diner avec un collègue plus
jeune et plus charismatique qui avait remporté le prix Nobel et qui
avait travaillé pour le projet Manhattan. Beaucoup ont suspecté qu'il
s'était suicidé avec les composés de cyanure sur lesquels il travaillait
mais le médecin légiste a conclu à une crise cardiaque, sans vraiment
chercher plus loin.

Tout ceci pour dire que la représentation de Lewis que nous utilisons
pour symboliser la façon dont les atomes se lient entre eux est quelque
chose qui a été inventé par un scientifique fou et torturé. Ce n'est pas
quelque chose d'abstrait qui a toujours existé. C'est un outil qui a été
pensé par quelqu'un et qui s'est avéré tellement pratique que nous
l'avons adopté.

\hypertarget{la-regle-de-loctet}{%
\section{La regle de l'octet}\label{la-regle-de-loctet}}

En biologie, la plupart des composés peuvent être représentés selon la
\emph{structure de Lewis.} Et voilà comment ça fonctionne : ces
structures montrent essentiellement comment les atomes se lient ensemble
pour faire des molécules. Et une des règles de base quand on fait ces
diagrammes est que les éléments avec lesquels on travaille réagissent de
telle façon à ce que chaque atome obtienne un total de 8 électrons sur
sa dernière couche. Et ça s'appelle \emph{la règle de l'octet.}

Parce que les atomes veulent compléter leur octet d'électrons pour être
heureux et satisfaits. L'oxygène a 6 électrons sur sa dernière couche et
en a besoin de 2 pour respecter la règle de l'octet, ce qui explique
pourquoi on obtient \chemform{H_2 O}. Il peut aussi se lier au
carbone qui en a besoin de 4. Donc on a 2 doubles liaisons avec 2 atomes
différents d'oxygène et on obtient du \chemform{C O_2}. Ce fichu gaz
à effet de serre et également la chose qui rend toute vie sur Terre
possible. L'azote a 5 électrons sur sa dernière couche électronique.
Voilà comment on les compte: Il y a 4 places à prendre. Chacune d'entre
elles veut 2 atomes. Et comme les gens qui prennent le bus, ils
préfèrent ne pas s'asseoir l'un à côté de l'autre. Je ne rigole pas, ils
ne veulent avoir personne à côté d'eux, jusqu'à ce qu'ils n'aient
vraiment plus le choix. Donc pour un bonheur maximal, l'azote se lie
avec 3 hydrogènes pour former l'ammoniac. Ou alors avec 2 hydrogènes qui
dépassent d'un autre groupe d'atomes, ce qu'on appelle un amine. Et si
cette amine est liée à un carbone qui est lié à un groupe d'acide
carboxylique alors on obtient un acide aminé !

\hypertarget{liaisons-covalentes-polaires-et-non-polaires}{%
\section{Liaisons covalentes polaires et non
polaires}\label{liaisons-covalentes-polaires-et-non-polaires}}

Tout le monde a déjà entendu parler d'eux, n'est-ce pas ? Parfois des
électrons sont partagés équitablement dans une liaison covalente comme
avec \chemform{O_2}. Ce qui s'appelle une liaison covalente
parfaite. Mais souvent, un des participants est plus cupide que l'autre.
Dans l'eau par exemple, la molécule d'oxygène aspire les électrons vers
elle et ils passent plus de temps avec l'oxygène qu'avec les hydrogènes.
Ceci crée une légère charge positive autour des hydrogènes et une légère
charge négative autour des oxygènes. Quand quelque chose est chargé, on
dit qu'il est polarisé. Il a un pôle positif et un pôle négatif. Et donc
c'est une liaison covalente polarisée.

\hypertarget{liaisons-ioniques}{%
\section{Liaisons ioniques}\label{liaisons-ioniques}}

Maintenant parlons brièvement d'un type de liaison complètement
différent, qu'est la liaison ionique. Elle se produit quand, à la place
de partager des électrons, les atomes donnent ou acceptent de bon coeur
les électrons d'un autre atome et puis vivent heureux en tant qu'atomes
chargés. Et les atomes chargés, ça n'existe pas en fait. Si un atome a
une charge, c'est un ion. Les atomes en général préfèrent être neutres,
mais s'ils ont moyen d'obtenir un octet complet, ça ne les dérange pas
tant que ça, les atomes vont parfois faire des sacrifices pour obtenir
cet octet. Le composé ionique le plus courant dans notre vie de tous les
jours est le sel. Chlorure de sodium. \chemform{Na Cl}. Ce truc,
malgré son goût délicieux, comme mentionné plus tôt est faits de deux
composants vraiment ``méchants''. Sodium et chlore. Le chlore est ce
qu'on appelle un halogène, un élément qui n'a besoin que d'un électron
pour obtenir l'octet. Et le sel est un métal alcalin ce qui veut dire
qu'il n'a qu'un électron sur sa dernière couche. Donc le chlore et le
sodium sont si proches de leur but, qu'ils détruiront volontier tout ce
qui se trouve en travers de leur route pour obtenir cet octet. Et par
conséquent, il n'y a pas de meilleure solution que de mettre le chlore
et le sodium ensemble et de les laisser s'aimer. Ils font immédiatement
le transfert d'électrons. Comme ça le sodium perd son électron en trop,
et le chlore obtient un octet. Ils deviennent Na+ and Cl- et sont
tellement chargés qu'ils collent ensemble. Et on appelle cette adhérenc
une liaison ionique. C'est juste pareil si vous avez deux amis
complètement fous, ça peut être intéressant de les mettre ensemble,
comme ça ils arrêtent de vous ennuyer. La même chose fonctionne avec le
chlore et le sodium. Vous les mettez ensemble, et ils n'ennuient plus
personne. Et soudainement, ils ne veulent plus tout détruire, ils
veulent juste être délicieux. Les changements chimiques comme celui-ci
sont impressionants.

\hypertarget{liaison-hydrogene}{%
\section{Liaison hydrogene}\label{liaison-hydrogene}}

Mais, plus haut le chlore et le sodium voulaient nous
tuer, et maintenant ils ont bon gout. Voilà, nous arrivons à la dernière
liaison dont nous allons parler dans notre introduction à la chimie et
c'est la liaison hydrogène. Imaginez que vous vous souvenez de l'eau,
j'espère que vous n'avez pas oublié l'eau. Puisque l'eau est "collée"
par une liaison covalente polarisée la partie avec l'hydrogène est
chargée positivement et la partie avec l'oxygène est chargée
négativement. Donc quand nos molécules d'eau se déplacent, généralement
on pense que c'est un fluide parfait, mais en réalité, elles collent un
peu les unes aux autres. L'hydrogène juste à côté de l'oxygène. Vous
pouvez même le voir de vos propres yeux, si vous remplissez trop un
verre d'eau, ça formera une bulle sur le dessus. L'eau va coller
ensemble sur le dessus. C'est l'oeuvre des liaisons covalentes
polarisées, qui collent les molécules les unes aux autres pour qu'elles
ne coulent pas en dehors du verre. Ces liaisons hydrogène relativement
faibles arrivent dans toute sorte de composés chimiques différents, pas
uniquement dans l'eau. Et en fait, il jouent un rôle très important dans
les protéines qui sont en quelque sorte les substances chimiques qui
composent l'entièreté de nos corps. Une dernière chose à retenir ici,
c'est que les liaisons, même les covalentes et les ioniques même dans
leur propre catégorie ont souvent des forces différentes. Et on a
l'habitude de juste les écrire avec une petite ligne mais cette ligne
peut représenter ou une liaison covalente très très forte ou une
relativement faible. Parfois des liaisons ioniques sont plus fortes que
des liaisons covalentes. mais généralement, ce n'est pas le cas, et la
force des liaisons covalentes varie grandement. La façon dont ces
liaisons se forment et se cassent est intensément important à la vie, et
à nos vies. Créer et casser des liens est en fait, la clé de la vie
elle-même et aussi la clé de la mort. Par exemple, si vous ingérez du
sodium.

Gardez bien ça à l'esprit au fur et à mesure que nous avancerons dans la
biologie : Même la personne la plus sexy que vous avez rencontrée dans
votre vie est juste une collection de composés organiques qui flottent
dans un sac d'eau.
