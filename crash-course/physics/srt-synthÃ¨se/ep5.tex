\documentclass[a4paper, twoside]{article}

%% Language and font encodings
\usepackage[french]{babel}
\usepackage[utf8x]{inputenc}
\usepackage[T1]{fontenc}


%% Sets page size and margins
\usepackage[a4paper,top=3cm,bottom=2cm,left=3cm,right=3cm,marginparwidth=1.75cm]{geometry}
\usepackage{fancyhdr}
\pagestyle{fancy}

%% Useful packages
\usepackage{amsmath}
\usepackage{graphicx}
\usepackage[colorinlistoftodos]{todonotes}
\usepackage[colorlinks=true, allcolors=blue]{hyperref}
\usepackage{float}


\title{MCU, Newton laws, gravitation}
\author{Jean Aboutboul}
% Update supervisor and other title stuff in title/title.tex

\begin{document}
\maketitle
\renewcommand{\headrulewidth}{0pt}
\strut

\hypertarget{les-lois-de-newton}{%
\section{Les lois de Newton}\label{les-lois-de-newton}}

Nous avons beaucoup parlé de la science derrière le déplacement des
objets - on lance une balle en l'air, il y a des méthodes pour prédire
exactement comment elle va tomber. Mais il y a quelque chose dont nous
n'avons pas parlé : les forces, et pourquoi elles font accélérer les
choses. Et pour ça, nous allons nous tourner vers un physicien dont tout
le monde a sûrement entendu parler : \emph{Isaac Newton}. Avec ses trois
lois, publiées en 1687 dans son livre \emph{Principia}, Newton a décrit
sa compréhension du mouvement -- et ses idées étaient novatrices.
Aujourd'hui, plus de 300 ans plus tard, si nous essayons de décrire les
effets de forces sur à peu près n'importe quel objet de tous les jours
-- une boîte par terre, un renne qui tire un traîneau, ou un ascenseur
qui nous amène à notre appartement -- nous aurons envie d'utiliser les
lois de Newton. Et oui. Je vais expliquer le truc du renne dans une
minute.

\hypertarget{principe-dinertie}{%
\subsection{Principe d'inertie}\label{principe-dinertie}}

La première loi de Newton parle d'inertie, qui est simplement la
tendance d'un objetà continuer de faire ce qu'il fait. Elle est souvent
décrite ainsi : \textbf{\emph{``Un objet qui se déplace restera en
déplacement, et un objet au reposrestera au repos, à moins qu'une force
agisse sur lui.}} Ce n'est rien d'autre qu'une autre manière de dire que
pour changer la manière dont quelque chose se déplace -- lui donner de
l'ACCELERATION -- nous aurons besoin d'une force résultante. Donc,
comment mesure-t-on l'inertie ? Eh bien, la chose la plus importante à
connaître est la masse. Disons que nous avons deux balles qui ont la
même taille, mais l'une est un ballon de plage et l'autre est une balle
de bowling. La balle de bowling sera plus dure à déplacer, et plus dure
à arrêter une fois qu'elle se déplacera. Elle a plus d'inertie car elle
a plus de masse. Plus de masse signifie plus de MATIERE, avec une
tendance à continuer ce qu'elle faisait avant que notre force arrive et
l'interrompe.

\hypertarget{f-ma}{%
\subsection{\texorpdfstring{\(F = ma\)}{F = ma}}\label{f-ma}}

Cette idée se relie bien avec la deuxième loi de Newton : la force
résultante vaut la masse fois l'accélération. Ou, sous forme d'équation,
\(F(res) = m*a\). Il est important de se rappeler que nous parlons de
force RESULTANTE ici -- la quantitéde force restante une fois que l'on a
additionné toutes les forces qui peuvent s'annuler les unes avec les
autres. Imaginons que nous avons un palet de hockey sur une patinoire
parfaite sans frottements. La glace n'est en effet pas sans frottement
mais faisons comme si. Si nous poussons le palet avec notre crosse,
c'est une force qui agit dessus -- qui n'est pas annulée par quoi que ce
soit. Le palet subit donc une accélération. Mais lorsque le palet reste
immobile, ou même lorsqu'il glisse après que nous l'ayons poussé, alors
toutes les forces s'équilibrent. C'est ce qu'on appelle une position
d'équilibre. Un objet à l'équilibre peut toujours se DEPLACER, comme le
palet qui glisse, mais sa VITESSE ne change pas. C'est lorsque les
forces se DESéquilibrent que nous commençons à voir des choses
intéressantes. Le cas le plus courant de force résultante faisant se
déplacer quelque chose est probablement\emph{la force gravitationnelle}.
Disons que nous lançons une boule de 5kg dans les airs. La force
gravitationnelle tire sur la boule, qui accélère vers le bas au rythme
de \(9, 81 m/s²\). Ainsi la force résultante est égale à \(m*a\), mais
la seule force agissant ici est la gravité. Cela signifieque si nous
pouvions mesurer l'accélération de la boule, nous pourrions connaître la
force gravitationnelle. Et nous POUVONS mesurer l'accélération -- c'est
\(9, 81 m/s²\), la valeur que nous avions appellépetit \(g\). Ainsi la
force gravitationnelle sur la boule doit être de 5 kg, qui est la masse
de la boule fois petit \(g\) ce qui revient à 49,05 kilogrammes fois
mètres par seconde carrée ! Nous utilisons cette équation pour la
gravité si souvent qu'elle est souvent écrite \(F(g) = m*g\). C'est
ainsi que l'on détermine la force gravitationnelle, aussi appellée
poids. Et ces unités peuvent être compliquées à dire, donc nous les
appellons simplement\emph{Newtons}. En effet, on mesure le poids en
Newtons -- en l'honneur de celui-ci -- et PAS en kilogrammes ! Les
kilogrammes sont une mesure de masse ! Mais la gravité n'est pas la
seule force agissant sur un objet. Donc lorsqu'on essaye de calculerune
force résultante, nous prenons généralement en compte plus que
simplement la gravité. C'est ici que nous allons parler d'une des forces
qui apparaît très souvent, qui est décritepar la troisième loi de
Newton.

\hypertarget{action---reaction}{%
\subsection{Action - reaction}\label{action---reaction}}

Nous connaissons probablement cette loi comme \emph{``Pour toute action
existe une réaction égale mais opposée''}. Ce qui signifie simplement
que si nous exerçons une force sur un objet, il exerce une force égale
sur nous. Et nous appellons cela la force \emph{normale}.
\emph{``Normal''} dans ce cas signifie \emph{``perpendiculaire''}, et la
force normale est toujours perpendiculaireà la surface sur lequel se
trouve l'objet. En tout cas, dans le cas où nous poussons un objet gros
et macroscopique, comme une table. Si nous posons un livre sur une
table, la force normale pousse -- et donc pointe -- verticalement. Mais
si nous le posons sur une table incliner, alors la force normale pointe
perpendiculairement à cette table. Cependant, la force normale n'est pas
comme les autres forces. Elle est particulière parce qu'elle change sa
magnitude. Disons que nous avons un petit bout de papier aluminium étiré
sur un bol, puis nous posons un petit grain de raisin dessus. A cause de
la gravité, le raisin exerce un peu de force sur le papier alu, et la
force normale pousse en réponse, de manière égale. Mais alors nous
ajoutons un autre raisin, ce qui double la force sur le papier alu --
dans ce cas, lan force normale double également. Cela continuera jusqu'à
ce que nous ayons mis suffisamment de raisins pour qu'ils transpercent
le papier alu. C'est ce qui arrive lorsque la force normale ne peut pas
équilibrer la force qui la pousse.

Mais que signifie réellement la troisième loi de Newton ? Lorsque
j'appuie sur mon bureau avec mon doigt, je lui applique une force. Et il
applique une force égale sur mon doigt -- une force que je peux
réellement sentir. Mais si c'est vrai -- et ça l'est -- alors pourquoi
pouvons-nous déplacer des objets ? \textbf{Simplement, les objets
bougent parce qu'il y a plus de choses qui se passent que juste des
forces d'action et de réaction.} Par exemple, lorsqu'un renne tire un
traîneau, la troisième loi de Newton nous dit que le traîneau tire
également avec une force égale. Mais le renne parvient à tirer le
traîneau parce qu'il est sur le sol. Lorsqu'il fait un pas, il pousse le
sol avec sa patte - et le sol le pousse vers l'AVANT. En même temps, le
renne tire également sur le traîneau, alors que le traîneau tire en
réponse. Mais la force du SOL qui POUSSE le renne vers l'avant est
SUPERIEURE à la force du traîneau qui le tire vers l'arrière. Donc
l'animal avance, et le traîneau fait de même.

\hypertarget{force-normale-force-de-tension-et-application}{%
\subsection{Force normale, force de tension et
application}\label{force-normale-force-de-tension-et-application}}

Maintenant que nous avons une idée des forces que nous pouvons
rencontrer, décrivons ce qui se passe lorsqu'une boîte se trouve par
terre. La première chose à faire -- \textbf{qui est la première chose à
faire SYSTEMATIQUEMENT lorsque, nous résolvons un problème de ce genre}
-- est l'isolement du solide. Tout simplement, on dessine vaguement les
contours de l'objet, on place un point en son milieu, et on trace et on
nomme des flèches représentant les forces. Il faut aussi choisir quelle
direction est positive -- dans ce cas, on choisit le haut comme positif.
Pour notre boîte, le dessin est simple. Il y a une flèche vers le bas,
représentant la force gravitationnelle et une flèche vers le haut,
représentant la force du sol sur la boîte. Comme la boîte ne bouge pas,
elle n'accélère pas, ce qui nous dit queces forces sont égales, donc la
force résultante vaut 0. Mais si nous attachons une corde au sommet de
cette boîte, et connectons cette corde au plafond pour que la boîte soit
suspendue dans les airs ? La force résultante est toujours nulle
puisqu'il n'y a pas d'accélération sur la boîte. Et la gravité tire
toujours vers le bas comme auparavant. Mais maintenant, la force qui
tire la boîte vers le haut vient de la CORDE, dans ce qu'on appelle la
force de tension. Pour simplifier nos exemples, on peut supposer que la
corde n'a pas de masse et est complètement indestructible -- peu importe
la force qui tire dessus, elle tirera en réponse. Cela signifie que la
force de tension n'est pas fixée. Si la boîte pèse 5 Newtons, alors la
tension de la corde vaut également 5 Newtons. Si on ajoute 5 Newtons de
poids à la boîte, la tension de la corde deviendra 10 Newtons. Un peu
comme la force normale changeait en ajoutant des raisins sur du papier
alu. Mais ici, c'est en réponse à une force de traction et non pas à une
poussée. L'astuce est que dans tous les cas, on peut additionner les
forces pour donner une force résultante -- même si la force résultante
n'est PAS toujours nulle. Un ascenseur par exemple. Disons que nous
sommes dans un ascenseur. La masse totale de l'ascenseur, nous inclus,
est de 1000 kg. Son déplacement est contrôlé par un contrepoids attaché
à une poulie. L'idée est de mettre un contre poids de 850 kg puis de
laisser l'ascenseur se déplacer. A ce moment-là, l'ascenseur descend -
parce qu'il est PLUS LOURD que son contrepoids. Avec l'espoir que le
contrepoids va l'empêcher d'accélérer de TROP. Mais comment savoir si ce
sera sécurisé ? A quelle vitesse l'ascenseur va-t-il descendre ? Pour le
savoir, isolons notre ascenseur en choisissant comme positif la
direction vers le haut. La force gravitationnelle tire l'ascenseur vers
le bas, et vaut la masse de l'ascenseur fois petit \(g\) - 9810 Newtons
de force dans la direction négative. Et la force de tension tire
l'ascenseur vers le haut, dans la direction positive. Cela veut dire que
pour l'ascenseur, la force résultante vaut la force de tension moins la
masse de l'ascenseur fois petit \(g\). Maintenant, étant donné que la
première loi de Newtons nous dit que \(F(rés) = m*a\), nous pouvons dire
que tout cela est égal à la masse de l'ascenseur fois l'accélération
vers le bas, - \(a\). C'est ce que nous cherchons à obtenir. Faisons la
même chose avec le contrepoids. La gravité le tire vers le bas avec
8338,5 N de force dans la direction négative. Ici encore, la force de
tension le tire vers le haut, donc la force résultante vaut la force de
tension moins la masse du contrepoids fois petit \(g\). Ici encore,
grâce à la deuxième loi de Newton, nous savons que tout cela vaut la
masse du contrepoids, fois la même accélération, ``\(a\)'' -- cette fois
positive puisque le contrepoids se déplace vers le haut. Donc ! Mettons
tout cela ensemble, nous avons deux équations et deux inconnues. Nous
n'avons pas de valeur pour la force de tension, ni pour l'accélération.
Mais nous cherchons à obtenir l'accélération. Nous utilisons pour cela
l'algèbre. Quand on a un système d'équations de la sorte, on peut
ajouter ou soustraire des termes de chaquecôté du signe égal pour en
faire une unique équation. Par exemple, si nous savons que \(1 + 2 = 3\)
et que \(4 + 2 = 6\), nous pouvons soustraire la première équation à la
deuxième pour obtenir \(3 = 3\). Dans notre cas, avec l'ascenseur,
soustraire la première équation de la deuxième nous débarasse du terme
représentant la force de tension. Maintenant, il nous reste à trouver
l'accélération -- c'est-à-dire que nous devons réarranger l'équationpour
obtenir tout le reste égal à ``\(a\)''. On obtient une équation qui dit
juste que ``\(a\)'' est égal à la différence entre les masses -- ou la
force résultante du système -- divisée par la masse totale. C'est
simplement une manière un peu plus stylisée d'écrire \(F(rés) = m*a\).
Et on peut calculer grâce à cela ``\(a\)'', qui vaut ici
\(0, 795 m/s²\). Ce qui n'est pas tant que ça d'accélération ! Donc à
priori, tant que nous ne descendons pas trop bas, ça devrait aller. Même
si l'arrivée est légèrement agitée.

\newpage

\hypertarget{mcu}{%
\section{MCU}\label{mcu}}

Avons-nous déjà participé à l'un de ces manèges tourbillonnants? nous
savons, ceux où nous entrons dans un cylindre géant et nous nous tenons
contre le mur, puis ils nous font tourner comme une salade humide ? Si
nous avons vécu cette expérience unique et nauséabonde, alors nous
savons que le simple fait de tourner en rond peut être\ldots{} intense.
Il se trouve également que c'est l'un des concepts les plus mal compris
de la physique newtonienne. C'est ce qu'on appelle un mouvement
circulaire uniforme, et c'est ce qui se produit lorsque quelque chose se
déplace le long d'une trajectoire circulaire de manière cohérente. La
plus grande partie de la confusion à propos de cette idée est liée au
fait que les choses accélèrent vers l'intérieur lorsqu'elles se
déplacent en cercle - une sorte d'accélération connue sous le nom
\emph{d'accélération centripète}. Mais nous entendrons souvent des gens
parler d'accélération centrifuge poussant les objets vers l'extérieur
lorsqu'ils se déplacent en cercle. C'est en fait là que les
centrifugeuses tirent leur nom ! Et l'accélération centrifuge n'est pas
fausse, exactement. C 'est juste pas réel. Donc, pour expliquer comment
les choses s'accélèrent vraiment lorsqu'elles tournent en rond, parlons
de la physique de ce trajet pendant qu'il nous fait tourner

En 1960, la NASA se préparait à envoyer des gens dans l'espace. Ils
savaient qu'une grande partie du vols patial impliquerait une
accélération, ils voulaient donc tester la quantité d'accélération que
les gens pouvaient supporter avant de s'évanouir. Parce que c'est ce qui
arrive quand trop de sang est expulsé de notre cerveau pendant trop
longtemps. Les ingénieurs ont donc testé des astronautes potentiels en
les mettant dans une centrifugeuse humaine - En gros, une version
surpuissante de ces manèges à la foire. Ils ont découvert que la plupart
des gens pouvaient supporter une accélération d'environ 98 mètres par
seconde au carré pendant 10 minutes. C'est environ dix fois
l'accélération causée par la gravité que nous ressentirions simplement
en sautant dans les airs.

Dans cet esprit, disons qu'on nous a demandé de calculer la sécurité de
l'un de ces manèges de carnaval -- ce qui signifie que nous devron
déterminer l'accélération que subiraient les coureurs. Il existe des
équations que nous pouvons utiliser pour ce faire, car, comme pour
beacoup d'autres types de mouvement, le mouvement circulaire uniforme a
quatre qualités principales : \textbf{la position, la velocité (ou
vecteur vitesse), l'accélération et le temps.} Et ils sont tous liés les
uns aux autres. Lorsqu'il s'agit d'un mouvement circulaire uniforme, la
position est la qualité la plus évidente~: il y a un objet, et il se
trouve sur une trajectoire circulaire. Mais la vélocité est un peu moins
intuitive. À tout moment, la velocité nous indique à quelle vitesse
l'objet va et dans quelle direction. Et cette direction n'est PAS le
long du chemin du cercle. C'est en fait perpendiculaire au rayon du
cercle - le long de ce que nous appelons une tangente. Donc, si nous
dessinons une flèche représentant la velocité sur le cercle, elle ne
touchera ce cercle qu'à un seul endroit. Cela peut sembler un peu
étrange, mais c'est vrai ! L'un des avantages de la physique du
mouvement est que, souvent, nous pouvons simplement l'essayer par
nous-même et voir ce qui se passe. Voici donc un moyen rapide de voir la
velocité tangentielle en action~: tout ce dont nous avons besoin est une
ficelle, une clé -- ou un autre petit objet auquel attacher la ficelle.
Déplacons la corde pour que la clé commence à tourner en cercles dans le
sens inverse des aiguilles d'une montre, parallèlement au sol. Ensuite,
lorsque la clé est au point du cercle le plus éloigné de nous lâchons la
ficelle. La clé vole vers la gauche ! C'est ce qu'on a vu grace aux lois
de Newton avec la force d'inertie et l'idée que si un objet est en
mouvement, il restera dans ce mouvement à moins qu'il ne soit sollicité
par une force externe nette. Ce qui signifie que quelque chose se
déplaçant en ligne droite va continuer à se déplacer en ligne droite à
moins qu'une force - qui n'est pas équilibrée par d'autres forces - ne
la fasse tourner. Chaque fois que nous voyons quelque chose tourner, il
y a une force externe nette qui agit dessus. C'est pourquoi, à un
instant donné, la velocité d'un objet se déplaçant dans un cercle lui
sera tangente. Sans force pour le faire tourner, il vole simplement dans
la direction dans laquelle il se déplaçait en dernier. Une fois que nous
avons lâché la corde, nous nous sommes débarrassé de la force qui
faisait tourner la clé en rond. Il a donc continué à se déplacer avec la
même vitesse qu'au moment exact où nous avons lâché prise -
perpendiculairement à la corde qui le relie à notre main, qui était le
centre de la trajectoire circulaire de la touche.

\hypertarget{force-centripede}{%
\subsection{force centripede}\label{force-centripede}}

Et maintenant, nous pouvons enfin parler de la force mystérieuse qui
accélérait la clé -- changeant la direction de sa vitesse pour qu'elle
se déplace en cercle. Cette force est la même raison pour laquelle les
cavaliers de la centrifugeuse humaine tournent en cercle - en fait,
c'est la raison pour laquelle tout bouge en cercle. Cette force est
connue sous le nom de \textbf{force centripète} et l'accélération
qu'elle provoque est appelée \textbf{accélération centripète}. La chose
importante à retenir à propos de l'accélération centripète est
qu'\textbf{elle est toujours dirigée vers le centre de la trajectoire
circulaire}. Cela a beaucoup de sens, si nous y réfléchissons en termes
de changement de vitesse. La clé ne faisait que tourner en rond parce
que notre main la tirait vers le centre d'un chemin circulaire. Mais
maintenant, pensons à ce que c'est que d'être sur l'un de ces tours de
centrifugeuse - ou, si nous ne nous en sommes jamais soumis, à ce que
c'est que d'être dans une voiture qui tourne brusquement. Le trajet - ou
la voiture - tourne e n cercle, il doit donc y avoir unea ccélération
centripète qui nous pousse vers le milieu d e ce cercle. Sauf que nous
avons l'impression d'être poussé v ers l'extérieur. Les gens attribuent
souvent cette sensationà la force centrifuge. Mais ce n'est pas réel. L
a raison pour laquelle les gens confondent laf orce centripète avec ce
qui ressemble à une f orce centrifuge se résume à un changement de
perspective- ce que les physiciens appellent un cadre de référence. À
partir du cadre de référence d'une personne se tenantà l'extérieur de la
centrifugeuse humaine, il est facile d e voir ce qui se passe réellement
: lorsque lec ylindre tourne, il oblige les personnes à l' i ntérieur à
se déplacer en cercle. Et le mur appuies ur eux pour les faire tourner
-- il les p ousse en fait vers le centre du cercle !M ais la personne à
l'intérieur du cylindre voit t out ce qui bouge avec elle.D 'après leur
cadre de référence, on a l'impression q u'ils sont juste écrasés contre
lem ur - comme si une f orce centrifuge agissait sur eux.M ais il n'y a
rien là pour réellement créer cette force. C'est pourquoi les physiciens
l'appellentu ne force fictive - elle n'existe pas vraiment. A lors!
Maintenant que nous savons comment fonctionne l'accélération lorsque
nous nous déplaçons en cercle, nous pouvons enfint rouver des moyens de
relier la position, la v itesse et l'accélération - et déterminers i ce
trajet en centrifugeuse est sans danger pour les personnes. M ais
d'abord, il faut parler du temps. Lorsqueq uelque chose se déplace
autour d'un cercle de m anière cohérente - en d'autres termes, son
accélératione st constante - il faudra un c ertain temps pour revenir à
ses conditions de départ.D ans ce cas, ces conditions de départ sont u n
point particulier le long de la trajectoire circulaire.N ous appelons ce
temps la période du mouvement, e t la variable que nous utilisons pour
le représenter estu n T majuscule. C e qui n'est pas trop difficile à
retenir, tant que nous gardons à l'esprit que la période est une d urée.
En chronométrant let our de la centrifugeuse en action, nous savons
qu'il faut 2 secondes p our tourner une fois. Nous dirions donc que lap
ériode de son mouvement est de 2 secondes. M ais parfois, il est plus
facile de parler del a même idée d'une autre manière~: combien de r
évolutions le véhicule effectue-t-il en une seconde~? C'estc e que nous
appellerions la fréquence du mouvement - que nous écrivons comme un f
dans les équations.C 'est assons simple à comprendre : s'il f aut 2
secondes au trajet pour faire un tour,a lors il fait un demi-tour p ar
seconde. Il n'est pas non plus trop difficile der elier la période et la
fréquence avec une équation : la f réquence est juste 1, divisé par la
période.M aintenant que nous avons gagné du temps, p arlons de position.
Nousp arlons généralement de distance en termes de circonférence d u
cercle, car cela nous indique combien de fois nous avons fait le tour du
cercle. E n d'autres termes, si un pilote de centrifugeuse couvrel a
même distance qu'une circonférence, nous s avons qu'il a fait un tour.
Et la circonférencee st juste 2 fois pi fois le rayon du c ercle. Donc,
si cette centrifugeuse humaine a unr ayon de 5 mètres, les coureurs
parcourraient 31,4 mètres à chaque révolution. M aintenant~: qu'en
est-il de leur vitesse~? Eh bien, dans nos é pisodes sur le mouvement en
ligne droite, nousa vons expliqué que la vitesse moyenne est
généralement é gale au changement de position au fil dut emps -- -- ce
qui s'avère être un excellent m oyen de décrire la vitesse d'un
mouvement circulaire uniforme.m ouvement. L orsque le cycliste a fait un
tour autourd u cercle, il a parcouru une distance égale à 2 fois pi fois
r -- ou, dans ce cas,3 1,4 mètres. C'est jusqu'où ils ont voyagé. E t le
temps qu'il a fallu était égal àl a période du mouvement du manège.
C'est l eur changement dans le temps.D ivisons la distance qu'ils ont
parcourue par l eur changement de temps, et nous obtenons l'é quation de
vitesse pour un mouvement circulaire uniforme. En utilisant c ette
équation, nous pouvons calculer la vitessed 'un cycliste sur la
centrifugeuse - c'est 15,7 m ètres par seconde.E nsuite, obtenir
l'équation de l'ampleur d e l'accélération centripète - quellee st sa
force, en gros - est un peu moins simple. C ette amplitude sera égale au
changementd e vitesse sur le changement de temps à un moment donné - en
d'autres termes, sa dérivée.E n fait, le calcul de la dérivée peut d
evenir compliqué, mais il s'avère être égalà la vitesse, au carré,
divisée par le rayon d u cercle.C ette équation a beaucoup de sens pour
plusieurs r aisons : Tout d'abord, jetons un œil aux unités.L
'accélération est mesurée en mètres par seconde au c arré, donc nous
savons déjà que quelle que soitl 'équation de l'accélération centripète
, les unités doivent être exprimées en mètres pars econde au carré. E t
ils le font : mettons la vitesse au carré, et nous nousr etrouvons avec
des unités de mètres au carré par seconde au c arré. Il suffit de
diviser ces unités par des mètrese t nous obtenons des mètres par
seconde au carré. nous p ouvons également dire à partir de cette
équation que si nous augmentons notre vitesse le long de la trajectoire
circulaire o u diminuons le rayon de cette trajectoire, nous devrions
nousr etrouver avec une accélération plus élevée. E t cette relation
entre l'accélération, lav itesse et le rayon se vérifie également dans
la vraie vie~: e ssayons de faire tourner la touche d'une corde plus
rapidemento u de raccourcir la corde mais de la faire tourner à l a même
vitesse. nous sentirons la touche tirerp lus fort sur nos doigts, car
elle subit p lus d'accélération.E t maintenant que nous avons une
équation pour l'accélération q ue les cyclistes subiraient sur la
centrifugeuse,n ous pouvons enfin déterminer si ce trajet est s ûr. On
sait déjà que leur vitesse serait de1 5,7 mètres par seconde, et que le
rayon d u trajet est de 5 mètres.A insi, selon l'équation de
l'accélération, l eur accélération serait de 49,3 mètres pars econde au
carré. C'est environ la moitié de l'accélération q ue la NASA a trouvée
ferait perdre connaissance aux gens.L e trajet est donc probablement
sûr, au moins p endant quelques minutes. Qu'une tellea ccélération soit
agréable est une autre histoire - mais b on, nous sommes juste là pour
nous assurer que la conduitee st sûre. Nous ne sommes pas responsables
de n ettoyer le vomi une fois qu'il est terminé.A ujourd'hui, nous avons
appris que lorsqu'un objet est e n mouvement circulaire uniforme, sa
vitesse estt angente au cercle et son a ccélération pointe vers
l'intérieur. Nous avons également parlé de lad ifférence entre les f
orces centripètes et centrifuges et des équations dérivées pour la
période, laf réquence, la vitesse et l'accélération. C rash Course
Physics est produit en associationa vec PBS Digital Studios. nous
pouvons nous r endre sur leur chaîne pour découvrir des émissions
incroyablesc omme Deep Look, The Good Stuff et PBS Space Time. C et
épisode de Crash Course a été filméd ans le studio Doctor Cheryl C.
Kinney Crash Course

\end{document}



