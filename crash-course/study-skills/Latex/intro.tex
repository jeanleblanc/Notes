\hypertarget{introduction}{%
\chapter*{Introduction}\label{introduction}}

Les informations ci-présentes seront tirées de cours donné par Thomas
Frank (fondateur du College Info Geek.com).

Ici, nous allons apprendre à apprendre. C'est une faculté importante à
acquérir et à parfaire ; vu que notre économie est de plus en plus
dominée par de l'information. Ceux qui retiennent ces informations,
apprennent rapidement des nouvelles techniques, et sauront allier leurs
connaissances de manière nouvelle et innovatrice, seront ceux qui
réussiront. De plus, lorsque nous revisons avec efficacité pour un
examen, il nous restera plus de temps pour une autre activité.

\emph{``Ces sept dernières années, j'ai passé beaucoup de temps à écrire
et faire des recherches sur la façon dont nous, en tant qu'êtres
humains, puissions améliorer notre capacité d'apprendre et de devenir
plus productifs. Je me suis entretenu avec des professeurs d'université,
des médecins spécialistes du sommeil, des neuroscientifiques et autres
experts. J'ai appris quelques astuces au fur et à mesure, qui vont
au-delà des bases de ce qu'on aurait pu vous déjà enseigner. Alors,je
partagerai ce que j'ai appris}{[}\ldots{]}'' (Thomas Frank)

Nous allons, notamment, approfondir sur la prise des notes avec plus
d'efficacité, faire face aux devoirs de lecture, et - bien sûr - réviser
pour des contrôles. Nous explorerons le fonctionnement de notre mémoire
et étudierons des stratégies d'étude qui tirent profit de ses limites,
au lieu d'aller à l'encontre. Nous allons aussi parler des techniques de
productivité comme la planification, lutter contre la procrastination et
améliorer notre concentration ; des outils qui vous serviront autant
dans la vie qu'en tant qu'étudiant, que sur le chemin que vous décidez
d'emprunter par la suite.
