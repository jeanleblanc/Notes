\hypertarget{prendre-des-notes}{%
\chapter{Prendre des notes}\label{prendre-des-notes}}

\hypertarget{pourquoi-se-concentrer-sur-vos-notes}{%
\section{Pourquoi se concentrer sur vos notes
?}\label{pourquoi-se-concentrer-sur-vos-notes}}

Eh bien, en termes simples, lorsqu'il s'agit d'apprendre et de retenir
des informations, la sortie est tout aussi importante que l'entrée ;
lorsque nous apprenons un fait ou un concept pour la première fois, nous
absorbons de nouvelles informations ; mais, pour conserver ces
informations pendant longtemps, nous devons les stocker dans un endroit
auquel nous pourrions facilement accéder plus tard, et nous devons les
mettre dans \textbf{vos propres mots.}

\hypertarget{outils-de-prise-de-note}{%
\subsection{Outils de prise de note}\label{outils-de-prise-de-note}}

Commençons par ce qui va nous amener vers le succès en premier lieu : se
présenter en classe préparé avec les bons outils. Il existe deux voies
que nous pouvons suivre lors de la sélection de ces outils : papier, ou
ordinateur. Quelle est la meilleure option ? 

Entre ces deux, le débat est en
cours depuis des années, mais nous avons des preuves scientifiques
récentes auxquelles nous pouvons nous tourner pour obtenir des réponses
concrètes. Selon une étude réalisée à l'Université de Princeton en 2014,
les étudiants qui ont pris des notes sur une conférence de 15 minutes à
l'aide d'un ordinateur portable ont écrit en moyenne 310 mots, tandis
que ceux qui ont écrit sur papier n'en ont écrit en moyenne que 173. Il
semble donc que prendre note à l'ordinateur donne définitivement un
avantage de vitesse. Cependant, ces mêmes étudiants étaient capables de
se souvenir de moins d'informations lorsqu'ils étaient testés plus tard.
Alors pourquoi cela arrive-t-il ? Eh bien, la racine du problème réside
dans le fait que les preneurs de notes informatiques étaient beaucoup
plus susceptibles de noter ce qui était présenté mot pour mot.
\textbf{tought bubble} Lorsque nous prêtons attention à une conférence,
il y a deux aspects de l'information auxquelles nous sommes exposé.
Étant donné que des informations complexes sont communiquées par le
langage - qu'il soit écrit ou parlé - nous obtenons à la fois la syntaxe
(comme les lettres et les sons qui composent les mots) ainsi que le
sens. Lorsque vous tapez vos notes, l'avantage de la vitesse vous permet
d'enregistrer une version beaucoup plus complète de ce que dit votre
professeur. Cependant, votre mémoire de travail - la partie de votre
mémoire qui traite les informations que vous recevez actuellement - ne
peut traiter qu'un nombre limité de choses à la fois. La recherche
actuelle en sciences cognitives évalue ce montant à environ quatre
``morceaux'' d'informations. La combinaison de cet avantage de vitesse
d'enregistrementet de votre limite de traitement mental intégrée peut
vous amener à consacrer plus de ressources mentales à la syntaxe du
message - ces lettres et sons embêtants - et moins à la signification
réelle. En conséquence, vous apprenez moins en classe et vous vous créez
plus de travail plus tard.

Alors, cela signifie-t-il qu'un stylo et du papier battent toujours
votre ordinateur portable ? Eh bien, pas nécessairement ; maintenant que
vous savez que l'augmentation de la vitesse que vous obtenez en tapant a
un inconvénient, vous pouvez simplement vous résoudre à taper moins et à
accorder plus d' attention à la signification du message pendant que
vous êtes en classe. Pourtant, le papier a un avantage implicite, car il
nécessite moins de maîtrise de soi. Votre vitesse d'écriture à la main
limite automatiquement votre attention à la syntaxe et, en prime, vous
n'avez pas non plus à vous soucier d'être tenté d'aller sur Youtube en
plein cours.

Quel que soit l'outil que vous décidez de choisir, assurez-vous d'être
préparé en classe Si vous utilisez du papier, ayez un cahier bien
organisé avec beaucoup d'espace vierge, ainsi qu'un stylo de bonne
qualité avec lequel vous aimez écrire. Et si vous décidez qu'un
ordinateur correspond mieux à votre style, trouvez une bonne application
de prise de notes comme Evernote, ou OneNote, Dropbox Paper, ou toute
autre qui correspond à votre fantaisie. Vous devez également fermer
toutes les applications ou sites Web qui ne sont pas pertinents pour la
conférence - cela vous aidera à rester concentré, même si vous devrez
peut-être encore travailler pour ignorer ce gars devant vous qui répond
à un quiz sur Buzzfeed pour pour savoir quelle est sa maison de
Poudlard. Je suis un Serdaigle, au fait - même si c'est exactement ce
qu'un Serpentard dirait, n'est-ce pas ? ;)

\hypertarget{quelle-information-noter}{%
\subsection{Quelle information noter ?}\label{quelle-information-noter}}

Quoi qu'il en soit, maintenant que vous êtes préparé et équipé des bons
outils, que devriez-vous exactement noter avec eux~? Après tout, vous ne
pouvez pas tout noter. Comme l'a noté le célèbre mathématicien
\emph{Eric Temple Bell}, \emph{``La carte n'est pas la chose
cartographiée.''} Tout comme une carte n'est utile que si elle résume et
simplifie ce qu'elle représente, vos notes ne sont un outil de révision
utile que lorsque le rapport signal sur bruit est élevé. Cela signifie
qu'ils doivent contenir les informations dont vous avez besoin pour les
tests et les applications ultérieures, et dépourvus de tout ce qui n'a
pas d'importance. Il est difficile de faire des recommandations
spécifiques ici, car il y a tellement de matières et de cours différents
dans lesquels vous aurez besoin de vos compétences en prise de notes ;
cependant, regardons quelques directives générales qui nous orienterons
dans la bonne direction. Tout d'abord, évaluez chaque cours que vous
suivez dès le début. Examinez attentivement le programme, faites
attention aux guides d'étude ou aux documents de révision sur lesquels
vous pouvez mettre la main, et prenez des notes mentales sur les
différents types de questions que vous voyez sur les premiers exercices
et tests. De plus, chaque fois que vous entendez votre professeur dire
quelque chose comme \emph{``C'est important, faites attention''}, en
cours, c'est un signal pour prendre des notes très prudentes. Beaucoup
de mes amis à l'école pensaient que c'était un signal pour faire une
sieste, mais ils avaient tort. Au-delà de cela, que vous soyez assis en
classe ou que vous fassiez un devoir de lecture dans votre manuel, vous
voudrez accorder une attention particulière à des éléments tels que~:
les grandes idées - vous savez, les résumés, les aperçus ou les
conclusions les listes à puces \textbf{(comme celle-ci ) Termes et
définitions et exemples} Les exemples sont doublement importants, en
particulier dans les cours où vous devez appliquer des concepts et des
formules à des problèmes, comme en mathématiques ou en physique. Vous
vous souvenez probablement des moments où un exemple présenté en classe
était parfaitement logique, mais un problème de devoirs ultérieur
utilisant exactement le même concept vous a complètement laissé
perplexe. Il y a une grande différence entre pouvoir suivre pendant que
quelqu'un d'autre résout un problème et avoir les côtelettes pour le
résoudre par vous-même. Mais en enregistrant chaque détail des exemples
que vous voyez en classe - ainsi qu'en prenant des notes sur les raisons
pour lesquelles les concepts utilisés fonctionnent - vous aurez beaucoup
plus de munitions pour travailler pendant que vous vous attaquez à ces
problèmes de devoirs.

Maintenant que nous avons couvert les éléments de bonnes notes utiles,
entrons dans les détails des mainière dont nous allons les prendre. Il
existe de nombreux systèmes de prise de notes, chacun avec ses propres
avantages et inconvénients, mais ici, nous allons nous concentrer sur
trois : la méthode \textbf{Outline}, la méthode \textbf{Cornell} et la
méthode \textbf{Mind-Mapping}.

La méthode Outline est probablement la plus simple de toutes, et c'est
probablement celle que vous connaissez le mieux. Pour l'utiliser, il
vous suffit d'enregistrer les détails de la conférence ou du livre que
vous lisez dans une liste à puces. Chaque point principal sera une puce
de niveau supérieur, et en dessous, vous indenterez de plus en plus au
fur et à mesure que vous ajouterez des détails et des spécificités. Le
programme que j'ai écrit pour cette synthèse est un bon exemple de notes
de style plan. Le plan de chaque chapitre comporte plusieurs puces de
niveau supérieur, suivies de plusieurs niveaux de détail. Et oui, c'est
mon plan réel. Maintenant, la méthode Outline est idéale pour créer des
notes bien organisées, mais comme elle est si rigide, vous pouvez
facilement vous retrouver avec une tonne de notes qui se ressemblent
toutes. Donc, pour éviter que cela ne se produise, utilisez des astuces
de formatage pour faire ressortir les détails importants lorsque vous
les examinerez plus tard. \emph{Par exemple}~: dans ces notes que j'ai
prises lors d'un cours sur les systèmes d'information, vous pouvez voir
que j'ai noté plusieurs détails sous ``Prototypage''. Tous étaient
suffisamment importants pour être écrits, mais comme le professeur a
spécifiquement mentionné que le développement rapide et le faible coût
étaient les aspects les plus importants du prototypage, je me suis
assuré de mettre cette ligne en gras.

Ensuite, la méthode Cornell. Développée par Walter Pauk, professeur à
l'Université Cornell, dans les années 1950 et popularisée dans son livre
Comment étudier à l'université, la méthode Cornell est un système
éprouvé qui consiste à diviser votre papier (ou un tableau dans votre
application de prise de notes) en trois sections distinctes~: la colonne
\emph{Cue}, la colonne \emph{Notes} et la colonne \emph{Summary}.
Pendant une conférence, prenez vos notes réelles dans la colonne
\emph{Notes} bien nommée. Nous verrons qu'il n'y a pas beaucoup
d'erreurs d'orientation en ce qui concerne les compétences d'étude. Quoi
qu'il en soit, ici, vous pouvez utiliser la méthode de votre choix, que
ce soit la méthode de contour standard dont nous venons de parler ou
quelque chose de plus flexible. En même temps, lorsque vous pensez à des
questions qui n'ont pas reçu de réponse -- ou qui seraient d'excellentes
indice pour une révision ultérieure -- notez-les dans la colonne
\emph{Cue}. Ces questions vous seront utiles lorsque vous parcourrez vos
notes à l'avenir, car elles vous indiqueront les informations les plus
importantes et vous aideront à encadrer votre réflexion. La zone de
résumé restera vide jusqu'à la fin de la session. Une fois ce moment
venu, prenez deux ou trois minutes pour parcourir brièvement les notes
que vous avez prises et les questions que vous avez écrites, puis
rédigez un résumé de 1 à 2 phrases des principales idées qui ont été
abordées. Cela sert d'examen initial, ce qui aide à consolider tout ce
qui a été présenté ici et à solidifier votre compréhension pendant que
tout est encore frais dans votre esprit.

Si aucune de ces deux méthodes ne vous convient, vous aimerez peut-être
la dernière méthode que nous allons couvrir, à savoir la cartographie
mentale. Les cartes mentales sont des diagrammes qui représentent
visuellement les relations entre des concepts et des faits individuels.
Comme les cartes de style contour, elles sont très hiérarchiques mais
c'est là que s'arrêtent les similitudes. Les notes de style contour sont
linéaires et se lisent un peu comme du texte normal, tandis que les
cartes mentales ressemblent davantage à des arbres ou à des toiles
d'araignées. Pour créer une carte mentale, vous écrivez le concept
principal en plein milieu de la page, puis vous vous diversifiez à
partir de là pour étoffer les détails. Cette méthode fonctionne très
bien sur papier, mais il existe également des applications comme Coggle
qui vous permettront également de créer des cartes mentales sur votre
ordinateur.

Quelle est donc la meilleure méthode ? Eh bien, c'est à vous de décider.
Je vous recommande d'essayer chacune d'entre elles et de faire vos
propres ajustements au fur et à mesure. N'oubliez pas non plus que
toutes les classes ne fonctionneront pas mieux avec exactement la même
méthode. Vos notes d'histoire seront probablement très différentes de
vos notes de mathématiques.
